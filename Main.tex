\documentclass[a4paper,11pt]{article}
\usepackage{amsmath, amssymb}
\usepackage{mathtools}
\usepackage{geometry}
\geometry{margin=2.5cm}
\usepackage{hyperref}
\usepackage{tcolorbox}
\usepackage{lmodern}
\usepackage{graphicx}

\title{Étude de la série \(\displaystyle \sum_{n=1}^{\infty} \frac{1}{n^2 + \alpha}\)}
\author{}
\date{}

\begin{document}
\maketitle

\section*{1. Convergence de la série}

On considère la série :
\[
\sum_{n=1}^{\infty} \frac{1}{n^2 + \alpha}
\]
où \( \alpha \in \mathbb{R} \). 

\subsection*{Cas général \(\displaystyle \sum \frac{1}{n^p + \alpha}\)}

Pour tout \( \alpha \in \mathbb{R} \) et \( p > 0 \), on a :
\[
\frac{1}{n^p + \alpha} \sim \frac{1}{n^p} \quad \text{quand } n \to \infty
\]
Donc la série converge si et seulement si \( p > 1 \) (par comparaison à la série de Riemann).

\medskip
\textbf{Conclusion :}
\[
\sum_{n=1}^{\infty} \frac{1}{n^p + \alpha} \text{ converge } \iff p > 1
\]

\section*{2. Expression de la somme pour \( p = 2 \)}

Pour \( \alpha > 0 \), on a la formule suivante :
\[
\sum_{n=1}^{\infty} \frac{1}{n^2 + \alpha}
= \frac{\pi}{2\sqrt{\alpha}} \cdot \coth(\pi \sqrt{\alpha}) - \frac{1}{2\alpha}
\]

\begin{tcolorbox}[colback=blue!5!white, colframe=blue!80!black, title=Formule à retenir]
\[
\boxed{
\sum_{n=1}^{\infty} \frac{1}{n^2 + \alpha}
=
\frac{\pi}{2\sqrt{\alpha}} \cdot \coth(\pi \sqrt{\alpha}) - \frac{1}{2\alpha}
\quad \text{pour } \alpha > 0
}
\]
\end{tcolorbox}

\section*{3. Étude des limites}

\subsection*{Quand \( \alpha \to 0^+ \)}

On utilise le développement :
\[
\coth(x) = \frac{1}{x} + \frac{x}{3} + o(x)
\]
avec \( x = \pi \sqrt{\alpha} \), d'où :
\[
\sum_{n=1}^{\infty} \frac{1}{n^2 + \alpha} \to \frac{\pi^2}{6}
\quad \text{quand } \alpha \to 0^+
\]

\subsection*{Quand \( \alpha \to +\infty \)}

On utilise :
\[
\coth(x) = 1 + 2e^{-2x} + o(e^{-2x})
\quad \Rightarrow \quad
\sum_{n=1}^{\infty} \frac{1}{n^2 + \alpha} \to 0
\quad \text{quand } \alpha \to +\infty
\]

\section*{4. Résumé visuel}

\[
\boxed{
\lim_{\alpha \to 0^+} \sum_{n=1}^{\infty} \frac{1}{n^2 + \alpha} = \frac{\pi^2}{6}
\quad ; \quad
\lim_{\alpha \to +\infty} \sum_{n=1}^{\infty} \frac{1}{n^2 + \alpha} = 0
}
\]

\end{document}

